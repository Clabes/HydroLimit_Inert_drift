%%%%%%%%%%%%%%%%%%%%%%% file template.tex %%%%%%%%%%%%%%%%%%%%%%%%%
%
% This is a general template file for the LaTeX package SVJour3
% for Springer journals.          Springer Heidelberg 2010/09/16
%
% Copy it to a new file with a new name and use it as the basis
% for your article. Delete % signs as needed.
%
% This template includes a few options for different layouts and
% content for various journals. Please consult a previous issue of
% your journal as needed.
%
%%%%%%%%%%%%%%%%%%%%%%%%%%%%%%%%%%%%%%%%%%%%%%%%%%%%%%%%%%%%%%%%%%%
%
% First comes an example EPS file -- just ignore it and
% proceed on the \documentclass line
% your LaTeX will extract the file if required
\begin{filecontents*}{example.eps}
%!PS-Adobe-3.0 EPSF-3.0
%%BoundingBox: 19 19 221 221
%%CreationDate: Mon Sep 29 1997
%%Creator: programmed by hand (JK)
%%EndComments
gsave
newpath
  20 20 moveto
  20 220 lineto
  220 220 lineto
  220 20 lineto
closepath
2 setlinewidth
gsave
  .4 setgray fill
grestore
stroke
grestore
\end{filecontents*}
%
\RequirePackage{fix-cm}
%
\documentclass{svjour3}                     % onecolumn (standard format)
%\documentclass[smallcondensed]{svjour3}     % onecolumn (ditto)
%\documentclass[smallextended]{svjour3}       % onecolumn (second format)
%\documentclass[twocolumn]{svjour3}          % twocolumn
%
\smartqed  % flush right qed marks, e.g. at end of proof
%

%
 \usepackage{mathptmx}      % use Times fonts if available on your TeX system
 
%
% insert here the call for the packages your document requires
%\usepackage{latexsym}
% etc.
%\usepackage{chngcntr}
%\usepackage{amsthm}
\usepackage{amsmath}
\usepackage{amsfonts}
\usepackage{pifont}
\usepackage{dsfont}
\usepackage{mathtools}
\usepackage{bbm}
\usepackage{mathrsfs}
\usepackage{graphicx}

\usepackage{showlabels}

%\setenumerate[1]{label=\thesection.\arabic*.}
%\setenumerate[2]{label*=\arabic*.}
%
% please place your own definitions here and don't use \def but
% \newcommand{}{}
\newcommand{\ex}{\mathbb{E}}
\newcommand{\prob}{\mathbb{P}}
\newcommand{\set}[1]{\left\{#1\right\}}
\newcommand{\e}{\epsilon}
\newcommand{\ds}{\displaystyle}
\newcommand{\N}{\mathbb{N}}
\newcommand{\Z}{\mathbb{Z}}
\newcommand{\R}{\mathbb{R}}
\newcommand{\Var}{\mathrm{Var}}
\newcommand{\wt}[1]{\widetilde{#1}}
\newcommand{\lra}{\longrightarrow}
\newcommand{\mb}[1]{\mathbb{#1}}
\newcommand{\mr}[1]{\mathrm{#1}}
\newcommand{\mc}[1]{\mathcal{#1}}
\newcommand{\mi}[1]{\mathit{#1}}
\renewcommand{\bf}[1]{\mathbf{#1}}
\newcommand{\ld}{\lambda}
\renewcommand{\d}{\delta}
\newcommand{\F}{\mathcal{F}}
\newcommand{\md}{\mathrm{d}}
\newcommand{\SL}{\sum\limits}
\newcommand{\BR}{\mathbb R}
\newcommand{\dist}{\overset{d}{=}}
\newcommand{\ol}[1]{\overline{#1}}
\newcommand{\p}{\partial}
\newcommand{\n}{{(n)}}
\newcommand*\interior[1]{#1^{\mathsf{o}}}
\renewcommand{\L}{\mb{L}}

%
%\theoremstyle{definition}
%\theorem{example}{Example}
% \theorem{remark}[thm]{Remark}
%\counter{thm}[section]
%\theorem{problem}{Question}
%\theorem{quest}{Question}

%\theorem{goal}{Goal}
%% \setcounter{thm}
%
%\theoremstyle{plain}
%\theorem{theorem}[thm]{Theorem}
%%\numberwithin{thm}{section}
%\theorem{definition}{Definition}
%\theorem*{conj*}{Conjecture}
%
%\theorem{remark}{Remark}
%\theorem{lemma}[thm]{Lemma}
%\theorem{prop}[thm]{Proposition}
%\theorem{cor}[thm]{Corollary}

% Insert the name of "your journal" with
 \journalname{Journal of Statistical Physics}
%
\begin{document}

\title{Propagation of chaos for systems of reflecting diffusions with inert drift%\thanks{Grants or other notes
%about the article that should go on the front page should be
%placed here. General acknowledgments should be placed at the end of the article.}
}
%\subtitle{Do you have a subtitle?\\ If so, write it here}

%\titlerunning{Short form of title}        % if too long for running head

\author{Clayton Barnes         \and
        Haya Kaspi %etc.
}

%\authorrunning{Short form of author list} % if too long for running head

\institute{F. Author \at
              first address \\
              Tel.: +123-45-678910\\
              Fax: +123-45-678910\\
              \email{fauthor@example.com}           %  \\
%             \emph{Present address:} of F. Author  %  if needed
           \and
           S. Author \at
              second address
}

\date{Received: date / Accepted: date}
% The correct dates will be entered by the editor


\maketitle

\begin{abstract}
We study systems of diffusions reflecting inside a sufficiently smooth domain $D \subset \R^d$ whose drift
depends on the reflection local time. The one particle was studied by Bass, Burdzy, Chen, and Hairer (2010). We show that a strong exchangeability property together with strong uniqueness of the limit implies a strong version of propagation of chaos. The collection of empirical processes converge to a non-linear reflecting heat equation whose drift depends globally on the boundary. Furthermore, existence and uniqueness of this PDE are demonstrated using stochastic methods.

%Insert your abstract here. Include keywords, PACS and mathematical
%subject classification numbers as needed.
\keywords{First keyword \and Second keyword \and More}
% \PACS{PACS code1 \and PACS code2 \and more}
% \subclass{MSC code1 \and MSC code2 \and more}
\end{abstract}

\section{Introduction and Main Results}
We study systems of diffusions reflecting inside a closed and bounded domain $D \subset \R^d$, with $C^2$ smooth boundary, whose drift
depends on the reflection local time. The case of one particle was studied in varying degrees of generality, by Knight \cite{Knight2001}, White \cite{white2007}, and Bass, Burdzy, Chen and Hairer \cite{bass2010stationary}.

\section{Construction: System of Diffusions with Inert Drift}
% In the proofs we will be considering existence and uniqueness statements in a weak sense before eventually demonstrating existence and uniqueness in a strong sense. Thus will have slightly different set-ups on our underlying probability space along the way, but here mention
Let $(\Omega, (\mathcal{F}_t)_{t \geq 0}, \prob)$ be a probability space supporting $n$ independent $\mc{F}_t$-adapted $\R^d$ Brownian motions $\{B_i(t) : t \in [0, T], 1 \leq i \leq n\}.$ Let $x_0$ be a point in the interior of $D$, and for each $x \in \partial D$ let $\eta(x) \in \R^d$ be the unit inward normal of $\partial D$ at $x.$ Given this setup, we consider a system of processes $(X_i^\n, L_i^\n), 1 \leq i \leq n$ such that almost surely:

\begin{flalign}
	\begin{split}\label{intro:SDE_system}
		X_i^\n(t) &:= x_0 + B_i(t) + L_i^\n(t) + \int_0^tK^\n(s)\, \md s \in D,\\
		K^\n(t) &:= \frac{1}{n}\int_0^t\sum_{i = 1}^n\eta\big(X_i^\n(s)\big)\md|L_i^\n|(s),\\
		L_i^\n(t) &= \int_0^t\eta\big(X_i^\n(s)\big)\md|L_i^\n|(s),
	\end{split}{}
\end{flalign}

for all $t \in [0, T]$,
where $t \rightarrow |L_i^\n|(t)$ is the continuous, non-decreasing, local-time of $X_i^\n$ on $\partial D$. See \cite{bass2010stationary} for more discussion on the boundary local-time.

The following lemma is an analytic result that is used for constructing reflecting Brownian motion (and other diffusions) inside a $C^2$-smooth domain such as $D$. If one replaces $f$ pathwise by a Brownian motion $B,$ the resulting process $x_B$ produced from the lemma below is Brownian motion reflecting inside $D,$ and $|\ell_B|$ is the corresponding boundary local time. Consequently, this gives us a strong solution and strong existence for the reflecting diffusions holds because the lemma can be applied to any probability space supporting a Brownian motion.

\begin{lemma}[Skorohod Map of Reflection, \cite{Lions_Sznitman}]\label{lemma:s_map}
	For any $f \in C([0, T], \R^d)$ with $f(0)$ in the interior of $D$, there is a unique pair of continuous functions $(x_f, \ell_f)$
	such that
	\begin{align}
		x_f(t) &= f(t) + \ell_f(t) \in {D}\label{eq:smap1}\\
		\ell_f(t) &= \int_0^t\eta(x_f(s)) \md |\ell_f|(s)\label{eq:smap2}
	\end{align}
	where $t \to |\ell_f|(t)$ is a non-decreasing continuous function whose generated measure is supported on 
	$\{t : x_f(t) \in \partial D\}.$ 
	That is, $|\ell_f|(t) = \int_0^t\mathds{1}(x_f(s) \in \partial D)\md |\ell_f|(s)$.
	Furthermore, the map $\Gamma : C([0, T], \R^d) \to C([0, T], \R^d) \times C([0, T], \R^d)$ defined by
	\[
	\Gamma(f) = \big(\Gamma_1(f), \Gamma_2(f)\big) := \big(x_f, \ell_f\big)
	\]
	is H\"{o}lder continuous of exponent $1/2$.
\end{lemma}

\begin{theorem}[Strong Existence and Weak Uniqueness]\label{th:existence_system}
	There exists a strong solution to \eqref{intro:SDE_system}.
\end{theorem}
\begin{remark}
	(Only strong existence and weak uniqueness are needed in the results of this paper. However, strong uniqueness can be proven by following a similar method as given in \cite[Section 3]{bass2010stationary}.)
	% in which case strong existence would follow from Yamada-Watanabe.

\end{remark}
\begin{proof}
	We first prove weak uniqueness using a Girsanov change of measure. Given our probability space $(\Omega, (\mc{F}_t)_{t \geq 0}, \prob)$, let
	\[
	\mathcal{E}(t) = \exp\left\{\sum_{i = 1}^n\int_0^tK^{(n)}(s)\md B_i(s) - \frac{n}{2}\int_0^t\left(K^{(n)}(s)\right)^2\md s\right\}
	\]
	be the local Girsanov transform where $(X_1^{(n)}, \dots, X_n^{(n)})$ is distributed as $n$ i.i.d.\ reflected Brownian motions under the measure $\md \mb{Q}_t = \mc{E}(t)\md \prob_t.$
It follows from \eqref{intro:SDE_system} and Lemma \ref{lemma:s_map} that $X_i^{(n)}$ is a process reflecting inside $D$ satisfying
	\[
	\big(X_i^{(n)}, L_i^{(n)}\big) = \left(\Gamma_1\Big(X_i(0) + B_i + \int_0^tK^{(n)}(s) \md s\Big), \Gamma_2\Big(X_i(0) + B_i + \int_0^tK^{(n)}(s) \md s\Big)\right),
	\] so H\"{o}lder continuity of the local time implies
	\begin{align}
		\begin{split}\label{eq:Gronwall}
			\|K^{(n)}\|_{[0, t]} &\leq \frac{1}{n}\sum_{i=1}^n\|L_i^{(n)}\|_{[0, t]}
			= \frac{1}{n}\sum_{i=1}^n\left\|\Gamma_2\big(X_i(0) + B_i + \int_0^tK^{(n)}(s) \md s\big)\right\|\\
			&\leq \frac{1}{n}\sum_{i=1}^n\left\|X_i(0) + B_i + \int_0^tK^{(n)}(s) \md s\right\|^{1/2}\\
			&\leq \frac{C}{n}\sum_{i=1}^n\|X_i(0) + B_i\|_{[0, t]}^{1/2} + \frac{C}{n}\sum_{i = 1}^n\left(\int_0^t\|K^{(n)}\|_{[0, s]}\md s\right)^{1/2}\\
			&= \frac{C}{n}\sum_{i=1}^n\|X_i(0) + B_i\|_{[0, t]}^{1/2} + C\left(\int_0^t\|K^{(n)}\|_{[0, s]}\md s\right)^{1/2}.
		\end{split}
	\end{align}
	Let 
	\[\alpha_n(t) :=  \frac{C}{n}\sum_{i=1}^n\|X_i(0) + B_i\|_{[0, t]}^{1/2},
	\] and 
	\[\beta_n(t) := \|K^{(n)}\|_{[0, t]}.\]
	Then we have
	\[
	\beta_n(t) \leq \alpha_n(t) + C\left(\int_0^t\beta_n(s) \md s\right)^{1/2}.
	\]
	By a non-linear version of the Gr\"onwall's inequality, or alternatively looking at a phase plane argument, there exists a constant $C'$ such that 
	\begin{align}\label{eq:bound_drift}
		\beta_n(t) \leq C'(t\alpha_n(t) + t)^2,
	\end{align}
	and so $\beta_n(t)$ does not blow up in finite time. From this, and standard stopping time arguments, uniqueness of the law given by $\prob_t$ follows from uniqueness of the law of $n$ i.i.d.\ Brownian motions reflecting inside $D$.
	\\
	\\
	\emph{(Strong Existence):}
	To show strong existence we construct a sequence of approximating processes that converge almost surely to a solution of \eqref{intro:SDE_system}. Our strong solution is motivated by the Skorohod construction of reflecting Brownian motion by defining a functional which we apply pathwise to a collection of i.i.d.\ Brownian motions. See \cite[Theorem 2.5]{barnes2020hydrodynamic}. Fix $\e > 0$ and let $f_i$, $1 \leq i \leq n$, be continuous functions from $[0, \infty)$
	to $D$ such that $f_i(0) = x_0.$ We recursively define the following functions in time-steps of size $\e.$ (We hide the dependence on $f$ and $n$ for ease of readability.) Set
	\begin{align}
		&I_\e(t) = 0, \,\text{for $t \in [0, \e)$},\\
		&(x_i^\e(t), \ell_i^\e(t)) = \Gamma(f_i + I_\e,t), \text{ for $t \in [(k-1)\e, k\e]$ and $1\leq i \leq k$},\\
		&I_\e(t) = I_\e(k\e) + \frac{1}{n}\sum_{i = 1}^n(t - k\e)\ell_i^\e(k\e), \text{ for $t \in [k\e, (k+1)\e]$ and $1 \leq i \leq n$}.
	\end{align}
	Intuitively, we are updating the drift at time steps of size $\e$ to be the average local-time amongst all particles at the previous step. From these definitions we see that
	\[
	I_\e(t) = \int_0^tK_\e(s)\md s
	\]
	where
	\begin{equation}
		K_\e(t) = \frac{1}{n}\sum_{i = 1}^n \ell_i^\e(\lfloor t/\e \rfloor \e)\label{eq:S_Existence3}.
	\end{equation}
	In particular,
	\[
	(x_i^\e, \ell_i^\e) = \Gamma\left(f_i + I_\e\right) = \Gamma\left(f_i + \frac{1}{n}\sum_{i = 1}^n\int_0^t\ell_i^\e(\lfloor s/\e \rfloor \e)\md s\right).
	\]
	Consequently, the same string of inequalities given in \eqref{eq:Gronwall} can be applied to $\|K_\e\|_{[0, t]}$. That is,
	
	\begin{align}
		\begin{split}\label{eq:Arzela_Ascoli_approx}
			\|K_\e\|_{[0, t]} &\leq \frac{1}{n}\sum_{i=1}^n\|\ell_i^{\e}\|_{[0, t]}
			= \frac{1}{n}\sum_{i=1}^n\left\| \Gamma_2\left(f_i + \int_0^tK_\e(s)\md s\right)\right\|\\
			&\leq \frac{1}{n}\sum_{i=1}^n\left\|f_i+ \int_0^tK_\e(s) \md s\right\|^{1/2}\\
			&\leq \frac{C}{n}\sum_{i=1}^n\|f_i\|_{[0, t]}^{1/2} + \frac{C}{n}\sum_{i = 1}^n\left(\int_0^t\|K_\e\|_{[0, s]}\md s\right)^{1/2}\\
			&= \frac{C}{n}\sum_{i=1}^n\|f_i\|_{[0, t]}^{1/2} + C\left(\int_0^t\|K_\e\|_{[0, s]}\md s\right)^{1/2}.
		\end{split}
	\end{align} 
	which shows that $\alpha(s) := \|K_\e\|_{[0, s]}$ satisfies the inequality 
	\[
	\alpha(s) \leq \frac{C}{n}\sum_{i=1}^n\|f_i\|_{[0, t]}^{1/2} + C\left( \int_0^t\alpha(s)\md s\right)^{1/2}.
	\] And as in the derivation of \eqref{eq:bound_drift}, there exists a bound, uniform in $\e$, for $\|K_\e\|_{[0, t]}.$ Therefore the collection $\{I_\e(\cdot) : \e > 0\} \subset C([0, T], \R^d)$ satisfies the Arzela-Ascoli 
	criterion. Hence there exists a subsequence $\e_k \to 0$ and an $I \in C([0, T], \R^d)$ such that
	$I_{\e_k} \lra I$ uniformly on $[0, T]$. By continuity of the map $\Gamma$ there exists $(x_i, \ell_i)$ such that
	\[
	(x_i^{\e_k}, \ell_i^{\e_k}) := \Gamma(f_i + I_{\e_k}) \lra \Gamma(f_i + I) =: (x_i, \ell_i), \text{for each $1 \leq i \leq n$}.
	\]
	From \eqref{eq:S_Existence3} 
	\[
	K_{\e_k} \lra \frac{1}{n}\sum_{i = 1}^n \ell_i, \text{uniformly on $[0, T]$}.
	\]
	In conclusion, the limits $(x_i, \ell_i, I), 1 \leq i \leq n,$ satisfy
	\begin{align}
		&(x_i, \ell_i) = \Gamma\Big(f_i + I\Big)\\
		&I(t) = \frac{1}{n}\sum_{i = 1}^n\int_0^t\ell_i(s)\md s.
	\end{align}
	Replacing each $f_i$ pathwise with independent Brownian motions $B_i$ yields the strong existence of solution \eqref{intro:SDE_system} given by $(X_i, L_i), i = 1, \cdots, n.$
\end{proof}

%NOTE Should I have two sections one with POC and the other with HL?
\section{Propagation of Chaos and Hydrodynamic Limit}
Intuitively, propagation of chaos holds for a collection of particles if every fixed finite number of particles become asymptotically independent. Typically we model this by a triangular array of processes, so that propagation of chaos means any finite fixed indexed collection
of processes becomes asymptotically independent as the number of particles approaches infinity. See \cite{Sznitman1}.  If the system is exchangeable, then the limit will be 
\begin{definition}\label{def:POC}
	Propagation of chaos holds for a triangular array of continuous processes $(X_i^{(n)}: 1 \leq i \leq n, n \in \N)$ if for any $k \in \N,$ and fixed indices $i_1, \dots, i_k$, there exists independent continuous processes $\wt{X}_1, \dots, \wt{X}_k$ such that
	\[
	(X_{i_1}^{(n)}, \dots, X_{i_k}^{(n)}) \overset{d}{\lra} (\wt{X}_1, \dots, \wt{X}_k).
	\]
\end{definition}

In this section we prove a strong propagation of chaos result described below.
\begin{theorem}[Propagation of Chaos]\label{th:POC}
	Let $(\Omega, (\mc{F}_t)_{t \geq 0}, \prob)$ be a probability space supporting a sequence of i.i.d.\ $\R^d$-Brownian motions $B_i, i \in \N.$ As given by Theorem \ref{th:existence_system} there exists a triangular array of strong solutions to \eqref{intro:SDE_system} given by $(X_i^{(n)} : 1 \leq i \leq n, n \in \N).$ For any fixed $k$,
	\[
	(X_1^{(n)}, \dots, X_k^{(n)}) \lra (\wt{X}_1, \dots, \wt{X}_k),
	\]
	almost surely on $C([0, T], D)$, where $\wt{X}_i$ are independent and identically distributed. Furthermore, 
	\begin{align}\label{sde:limit}
		\begin{split}
			\wt{X}_1(t) &= x_0 + B_1(t) + \wt{L}_1(t) + \int_0^t\ex \wt{L}_1(s)\md s,\\
			\wt{L}_1(t) &= \int_0^t\eta(\wt{X}_1(s)) \md|\wt{L}_1| s, 
		\end{split}
	\end{align}
	where $|\wt{L}_1|$ is the local time of $\wt{X}_1$ on $\partial D.$
\end{theorem}
\noindent
The proof of Theorem \ref{th:POC} will use two main ingredients: a strong exchangeability condition for the (common) drift of the $X_i$'s, and strong uniqueness of the solution to the limiting process \eqref{sde:limit}. Essentially,
\[
\text{(Strong Exchangeability) + (Strong Uniqueness of Limit) $\implies$ (Propagation Of Chaos).}
\]
% Continuity properties of the Skorohod map of reflection in Lemma \ref{lemma:s_map} will also be used throughout.
\begin{definition}
	Let $\{Y_i : 1 \leq i  \leq n\}$ be a collection of (perhaps $\R^d$-valued) random variables on a probability space $(\Omega, \prob)$. A random variable $Z$ adapted to $\sigma(Y_1,\dots, Y_n)$ is \emph{strongly exchangeable} if
	there is a measurable function $F$ such that $Z = F(Y_1, \dots, Y_n)$, and 
	$F(x_1, \dots, x_n) = F(x_{\sigma(1)}, \dots, x_{\sigma(n)})$ for any permutation $\sigma \in \mathbb{S}_n.$
\end{definition}
\noindent
One corollary of the construction of our strong solution for the system \eqref{intro:SDE_system} given in Theorem \ref{th:existence_system} is that there exists a strong solution of the processes $(x_i^{(n)}, L_i^{(n)}, K^{(n)})$ where
$K^{(n)}$ is strongly exchangeable with respect to the Brownian motions $(B_1, \dots, B_n).$
\begin{corollary}\label{cor:K_exchangeable}
	Let $(\Omega, (\mc{F}_t)_{t \geq 0}, \prob)$ be a probability space supporting a sequence of independent $d$-dimensional Brownian motions $B_i, i \in \N$, and $\big((X_i^{(n)}, L_i^{(n)}) : 1 \leq i \leq n, n \in \N\big)$ be the triangular array of processes defined on the same space as given in Theorem \ref{th:existence_system}. Then $K^{(n)}$ is strongly exchangeable with respect to the Brownian motions $(B_1, \dots, B_n)$ for every $n \in \N.$
\end{corollary}

\begin{proposition}[Strong Existence and Uniqueness of Limiting Process]\label{Prop:SDE_EU}
	Strong existence and uniqueness holds for solutions to the system 
	\begin{align}
		X(t) &= x_0 + B(t)  + L(t) + \int_0^t\ex L(s)\md s,\label{eq:SDE_Limit1}\\
		L(t) &= \int_0^t \eta(X(s)) \md|L|(s)\label{eq:SDE_Limit2},
	\end{align}\eqref{sde:limit}
	where $t \to |L(t)|$ is the local time of $X$ on $\partial D$.
\end{proposition}
\begin{remark}
	Our proof is similar to, and based on, the proof of path-wise uniqueness of Brownian motion with inert drift given in \cite{bass2010stationary}.
\end{remark}
\begin{proof}
Here we can write the reflection term $L(\cdot)$ as $\Gamma_2\left(B(\cdot) + \int_0^\cdot \ex L(s) \md s\right).$ First notice that under the given conditions, weak uniqueness of the pair $(X, L)$ implies strong uniqueness. This is because weak uniqueness implies $\ex L(t) =: g(t)$ is unique, and then $\eqref{eq:SDE_Limit1}-\eqref{eq:SDE_Limit2}$ become 
	\begin{align}
		X(t) &= x_0 + B(t)  + L(t) + \int_0^tg(s)\md s \in D,\\
		L(t) &= \int_0^t \eta(X(s)) \md|L|(s).
	\end{align}
	That is, $X$ is then reflected Brownian motion with a drift of $g$ inside $D$, which has a strong solution.
Existence is shown in the proof of Theorem \ref{th:POC} below.
\end{proof}

\noindent
Notice that \eqref{eq:bound_drift} in the proof of Theorem \ref{th:existence_system} gives a uniform (in terms of $n$) upper bound on the drift term $K^{(n)}$, which we state as a separate lemma below.

\begin{lemma}[Uniform bound on drift]\label{lemma:uniform_bound}
	There is a constant $C' > 0$ such that for all $n$, 
	\begin{align}\label{eq:lemma_bound_drift}
		\|K^{(n)}\|_{[0, t]} \leq C'\left(\frac{t}{n}\sum_{i = 1}^n\|X_i(0) + B_i\|_{[0, t]}^{1/2} + t\right)^2.
	\end{align}
\end{lemma}

%\begin{lemma}\label{lemma:K_continuity}
%For every $\d > 0$,
%\[
%|K^{(n)}(t) - K^{(n)}(t + \d)| \leq \frac{1}{n}\sum_{i =1}^n|B_i
%\]
%\end{lemma}

We now prove the propagation of chaos given the above results.

\subsubsection*{Proof of Theorem \ref{th:POC} (Propagation of Chaos)}
Let $(\Omega, (\mc{F}_t)_{t \geq 0}, \prob)$ be a probability space supporting a sequence of independent standard $d$-dimensional Brownian motions $B_i, i \in \N.$ We first analyze $K^{(n)}\in C([0, T], \R^d)$ by breaking it up into its components 
\[
K^{(n)}= \sum_{i = 1}^d K^{(n)}_i\mi{e}_i
\]
where the $\{\mi{e}_i : i = 1, \dots, d\}$ is the collection of standard basis vectors. We show each $K_i^{(n)}$ converges to a deterministic function
%\[
%\limsup_{n \to \infty} K_i^{(n)} = \liminf_{n \to \infty} K_i^{(n)}
%\]
by demonstrating that $\limsup K_i^{(n)}$ and $\liminf K_i^{(n)}$ have the same almost sure limit for each $i = 1, \dots, d.$ To do this we use the strong exchangeability for a finite system together with the Hewitt-Savage zero-one law. We begin by treating the base case of $i = 1$. Let 
$N_k^1$ be a random subsequence such that 
\begin{align}\label{eq:N_k}
	\lim_{k \to \infty} K_1^{(N_k^1)} = \limsup_{n \to \infty} K_1^{(n)} =: \ol{K}_1
\end{align}
almost surely. 
%We can choose $N_k^1$ to be strongly exchangeable because the sequence $K_1^{(n)}$ is strongly exchangeable. In other words, for a given $\omega = (\omega_1, \omega_2, \dots) \in \Omega$  representing the Brownian paths $(B_1, B_2, \dots)$, we know $K_1^{(n)}(\omega_\sigma) =  K_1^{(n)}(\omega)$ for large enough $n$, a.s., where $\omega_\sigma := (\omega_{\sigma(1)}, \omega_{\sigma(2)}, \dots)$ is the sequence of paths with indices permuted by a given permutation $\sigma.$ Hence we can take $N_k^1(\omega_\sigma) = N_k^1(\omega)$ a.s. for all such permutations $\sigma$ on finite indices. 

There exists a subsequence of $N_k^1$, call it $N_k^2$, such that
\[
\lim_{k \to \infty} K_2^{(N_k^2)} = \limsup_{k \to \infty} K_2^{(N_k^1)},
\]
almost surely.
Inducting over the indices of the coordinates, we find subsequences
$N_k^1, N_k^2, \dots, N_k^d$ such that \eqref{eq:N_k} holds, and in addition
\[
\lim_{k \to \infty} K_i^{(N_k^i)} = \limsup_{k \to \infty} K_i^{(N_k^{i -1})}
\]
for $1 < i \leq d.$ Furthermore, the last subsequence $N_k^d$ has the property that $K_i^{(N_k)}$ has an almost sure limit for each $i = 1, \dots, d.$ In particular, the limit of the first coordinate converges to its original limit superior. 
%That is,
%\[
%\limsup_{k \to \infty}K_1^{(N_k^d)} = \ol{K}_1,
%\]
%almost surely. 
Define
\begin{align}\label{eq:limit_K}
\ol{K} = \lim_{k \to \infty}K^{(N_k^d)} = \lim_{k \to \infty}\sum_{i = 1}^dK_i^{(N_k^d)}\mi{e}_i,
\end{align}
and notice $\ol{K}$ is strongly exchangeable with respect to the sequence of i.i.d.\ Brownian motions $B_1,B _2, \dots$. To see this, let $\omega = (\omega_1, \omega_2, \dots) \in \Omega$ be a representation of a particular sequence of the Brownian paths $\big(B_1, B_2 \dots\big)$, and a finite permutation $\sigma \in \mathbb{S}_N$ for a fixed $N \in \N.$ Define $\omega_\sigma := (\omega_{\sigma(1)},$ $\omega_{\sigma(2)}, \dots)$ as the sequence of paths with indices permuted by $\sigma.$ By construction and Corollary \eqref{cor:K_exchangeable}, we know
 \[
 K_i^{(n)}(\omega_\sigma) =  K_i^{(n)}(\omega), \, a.s.
 \]
 for all $n \geq N$ and $1\leq i \leq d$. Consequently, it follows from $\eqref{eq:limit_K}$ that 
 \[
 \ol{K}(\omega) = \lim_{k \to \infty} K^{(N_k^d)}(\omega) = \lim_{k \to \infty} K^{(N_k^d)}(\omega_\sigma) = \ol{K}(\omega_\sigma)
 \] for almost every $\omega$. That is, $\ol{K}$ is strongly exchangeable with respect to the sequence of i.i.d.\ Brownian motions $(B_i : i \in \N)$. 
%ecause $K^{(n)}$ and $N_k^d$ are both strongly exchangeable. 
By the Hewitt-Savage zero-one law, $\ol{K}$ is in fact a deterministic function. It follows from Lemma \ref{lemma:uniform_bound} that each coordinate of $K^{(N_k)}$ is uniformly bounded in $k$, almost surely, and consequently the bounded convergence theorem implies
\[
\left\|\int_0^\cdot K^{(N_k^d)}(s) \md s - \int_0^\cdot\ol{K}(s) \md s\right\|_{[0, t]} \lra 0,
\]
almost surely, as $k \to \infty.$ 

In other words, the drift $\ds \int_0^\cdot K^{(N_k^d)}(s)\md s$ converges a.s.\ to the (deterministic) continuous function $\ds \int_0^\cdot \ol{K}(s) \md s.$ We now define a coupling for a collection of reflected process with this deterministic drift. For each $i$, define
\begin{align}
	\wt{X}_i := \Gamma_1\left(X_i(0) + B_i + \int_0^\cdot \ol{K}(s)\md s\right),
\end{align}
so that
\[
\wt{X}_i(t) = X_i(0) + B_i(t) + \int_0^t \ol{K}(s) \md s + \wt{L}_i(t)
\]
where 
\begin{align}\label{eq:X_tilde}
	\wt{L}_i = \Gamma_2\left(X_i(0) + B_i + \int_0^\cdot \ol{K}(s)\md s \right)
\end{align}
is the reflection local time of $\wt{X}_i$ on $\partial D.$ So $\wt{X}_i$ is an approximation of $X_i^{(N_k^d)}$  where the drift of $X_i^{(N_k^d)}$ approaches the drift of $\wt{X}_i$ (for each $i$). By H\"older continuity of $\Gamma$, we have
\begin{align}
	\begin{split}\label{eq:LT_approx}
		&\left\|K^{(N_k^d)} - \frac{1}{N_k^d}\sum_{i = 1}^{N_k^d}\wt{L}_i\right\|_{[0, t]} \leq \frac{1}{N_k^d}\sum_{i = 1}^{N_k^d}\|\wt{L}_i - L_i^{(N_k^d)}\|_{[0, t]}\\
		&= \frac{1}{N_k^d}\sum_{i = 1}^{N_k^d}\left\|\Gamma_2\left(X_i(0) + B_i + \int_0^\cdot \ol{K}(s)\md s \right) - \Gamma_2\left(X_i(0) + B_i + \int_0^\cdot K^{(N_k^d)}\md s \right)\right\|_{[0, t]}\\
		&\leq \frac{C}{N_k^d}\sum_{i = 1}^{N_k^d}\left\|\int_0^\cdot \ol{K}(s) \md s - \int_0^\cdot K^{(N_k^d)}(s)\md s\right\|_{[0, t]}^{1/2}\\
		&\lra 0,
	\end{split}
\end{align}
almost surely. On the other hand, it is clear that $(\wt{X}_i : i \in \N)$ is a collection of independent processes because $\wt{X}_i$ is a function of $B_i$ and $\ol{K}$, and $\ol{K}$ is deterministic. Likewise,
$(\wt{L}_i : i \in \N)$ is an collection of independent processes. The strong law of large numbers implies that
\[
\frac{1}{n}\sum_{i = 1}^n \wt{L}_i(t) \lra \ex \wt{L}_1(t),
\]
almost surely. It follows from \eqref{eq:LT_approx} that 
\[
\int_0^\cdot \ol{K}(s) \md s = \int_0^\cdot \ex \wt{L}_1(s)\md s,
\]
almost surely, as well. This means the equation \eqref{eq:X_tilde} for $\wt{X}_i$ becomes
\[
\wt{X}_i(t) = X_i(0) + B_i(t) + \wt{L}_i(t) + \int_0^t\ex\wt{L}_i(s)\,\md s.
\]
By Proposition \ref{Prop:SDE_EU} we have (strong) uniqueness of the above equation, and in particular
we know that $\ex \wt{L}_i(t) = \ex \wt{L}_1(t)$ is a well defined continuous function in $\R^d.$ In other words,
\[
\ex \wt{L}_1 = \ol{K} = \lim_{k \to \infty} K^{(N_k^d)}.
\]
In particular, the first coordinate function of $\ol{K}$ is the first coordinate function of $G.$ That is,
\begin{align}\label{eq:project_1}
\pi_1\big(\ex \wt{L}_1(t)\big) = \ol{K}_1(t) = \limsup_{n \to \infty}K_1^{(n)}(t),
\end{align}
for all $t \in [0, T]$, almost surely,
where $\pi_i: \R^d \to \R$ is the projection onto the $i$th coordinate. \\
\indent The argument until this point can be repeated with the limit superior replaced by limit inferior in the definition of $\ol{K}$, which will give
\begin{align}\label{eq:project_2}
\liminf_{n \to \infty} K_1^{(n)} = \pi_1\big(\ex \wt{L}_1(t)\big),
\end{align}
almost surely. By \eqref{eq:project_1} and \eqref{eq:project_2} the limit superior and limit inferior of the first coordinate agree almost surely, and in particular
\[
\lim_{n \to \infty}K_1^{(n)}(t) = \pi_1\big(\ex \wt{L}_1(t)\big)
\]
almost surely. This shows the existence of $\lim_{n \to \infty}K_1^{(n)}.$ Inducting over the other coordinates $i = 2, \dots, d$, we see 
\[
\lim_{n \to \infty}K_i^{(n)}(t) = \pi_i\big(\ex \wt{L}_1(t)\big),
\]
for all $t \in [0, T]$, almost surely. So that
\begin{align}\label{eq:limit_K}
\lim_{n \to \infty}K^{(n)} = \ex \wt{L}_1(t),
\end{align}
almost surely, where $\ex \wt{L}_1(t)$ is the expected local time of the solution given in the \eqref{sde:limit}.
From \eqref{eq:limit_K} and continuity of $\Gamma$,
\begin{align*}
	X_i^{(n)} &= \Gamma_1\left(X_i(0) + B_i  + \int_0^\cdot K^{(n)}(s) \md s\right)\\
	&\overset{a.s.}{\lra} \Gamma_1\left( X_i(0) + B_i + \int_0^\cdot \ex \wt{L}_1(s) \md s\right)\\
	&= \wt{X}_i.
\end{align*}
This completes the proof of the propagation of chaos, and completes the existence portion of Proposition \ref{Prop:SDE_EU}.

\section{Hydrodynamic Limit}

If propagation of chaos holds for a triangular array, and if the system is exchangeable, then the empirical 
measure $\pi^{(n)}$ of the system with $n$ particles will converge to the probability measure concentrated on the law induced by the limit process of a single particle. The limit of the empirical measures (if it is unique) is called the hydrodynamic limit. This holds when we view the empirical measure
\[
\pi^{(n)}(\omega) := \frac{1}{n}\sum_{i = 1}^n\d_{X_i^{(n)}(\omega)}
\]
as a map from $\Omega$ to $\mc{P}(C([0, T], \R^d)),$ the space of probability measures on $C([0, T], \R^d)$
equipped with the metric of weak convergence.
\begin{proposition}[POC implies Hydrodynamic Limit]\label{prop:POCHyd}
	Let $(X_i^{(n)}: 1 \leq i \leq n, n \in \N)$ be a triangular array of continuous processes such that for each $n$, $(X_1^{(n)}, \dots, X_n^{(n)})$ is exchangeable. That is, $(X_1^{(n)}, \dots, X_i^{(n)}) \dist (X_{\tau(1)}^{(n)}, \dots, X_{\tau(n)}^{(n)})$ for any permutation $\tau \in \mathbb{S}_n.$ If propagation of chaos holds for $(X_i^{(n)}: 1 \leq i \leq n, n \in \N)$ then 
	\begin{align}\label{eq:Limit}
		\pi^{(n)}  \overset{d}{\lra} \d_{\wt{X}_1},
	\end{align}
	where $\wt{X}_1$ is the limiting process described in Definition \ref{def:POC}, and where $\d_{\wt{X}_1} \in \mc{P}(C([0, T], \R^d))$ is the (random) measure concentrated at $\wt{X}_1.$
\end{proposition}
\begin{remark}\label{remark:weakConv}
	The element $\d_{\wt{X}_1}$ is a random element in the space $\mc{M} := \mc{P}(C([0, T], \R^d))$ in the sense that $\omega \mapsto \d_{\wt{X}_1(\omega)}$ is a map from $\Omega$ to $\mc{P}(C([0, T], \R^d))$. We equip the space $\mc{M}$ with the metric $d$ of weak convergence, so $\pi^{(n)}$ and $\d_{\wt{X}_1}$ are $(\mc{M}, d)$ valued random elements. The limit in \eqref{eq:Limit} simply means convergence in distribution as random elements of the metric space $(\mc{M}, d).$
\end{remark}

\begin{proof}[Proof Sketch of Proposition \ref{prop:POCHyd}]
	Let $(\mc{M}, d)$ be the space of probability measures on $C([0, T], \R^d)$ with the metric of weak convergence, as given in Remark \ref{remark:weakConv}. For a distribution $Q$  on $\mc{M}$, consider $Q$ acting on a bounded measurable function $F : C([0, T], \R^d) \to \R$ by
	\[
	\langle Q, F \rangle := \ex_Q \left(\int F(\omega) \md \mu (\omega)\right) = \int_{\mc{M}} \int F(\omega) \md \mu (\omega) \, \md Q( \mu).
	\]
	For a given $F$, exchangeability implies
	\begin{multline}
		\langle \pi^{(n)} - \d_{\wt{X}_1}, F\rangle^2 \\ = \frac{1}{n^2}\ex F(X_1) + \frac{n-1}{n}\ex \big(F(X_1^{(n)})F(X_2^{(n)})\big)
		-2\ex F(X_1^{(n)}) \ex F(\wt{X}_1)  + [\ex F(\wt{X}_1)]^2.
	\end{multline}
	Propagation of chaos implies $(X^{(n)}_1, X^{(n)}_2) \lra (\wt{X}_1, \wt{X}_2)$ where $\wt{X}_i, i =1, 2,$ are independent. By exchangeability $\wt{X}_1 \dist  \wt{X}_2.$ Hence $ \frac{n-1}{n}\ex \big(F(X_1^{(n)})F(X_2^{(n)})\big) \to [\ex F(\wt{X}_1)]^2$, and consequently
	\[
	\langle \pi^{(n)} - \d_{\wt{X}_1}, F\rangle^2 \lra 0.
	\]
	That is, $\langle\pi^{(n)}, F\rangle \lra \langle \d_{\wt {X}_1}, F \rangle$ for every bounded and measurable $F$.
	%This says that $\pi^{(n)}$ converges to the constant 
	\qed
\end{proof}

There is another natural way of viewing $\pi^{(n)}$. For fixed $t,$ clearly 
\[
\pi^{(n)}_t := \frac{1}{n}\sum_{i = 1}^n\d_{X_i^{(n)}(t)}
\]
is a random measure with support in $D.$ That is, $\pi^{(n)}_t$ is a random element of $\mc{P}(D)$ equipped with
the metric of weak convergence. Letting $t$ vary within the interval $[0, T]$, continuity of the processes $X_i^{(n)}(\cdot)$ implies that
$\{\pi^{(n)}_t : t \in [0, T]\}$ is almost surely a continuous $\mc{P}(D)$-valued process.
We now have two ways of viewing $\pi^{(n)}$. First, as a map from $\Omega$ to $\mc{P}(C([0, T], D))$ as given in the statement of Proposition \ref{prop:POCHyd}. We call this scheme (A), and we call $\pi^{(n)}$ the \emph{empirical measure}. Second, as just mentioned, we can view $\pi^{(n)}_t, t \in [0, T],$ as a $\mc{P}(D)$-valued process, which we call scheme (B), and we call $\pi^{(n)}_t$ the \emph{empirical process}.
One can often show that convergence of the empirical measure in (A) implies convergence of the empirical process in (B) to the density of the limiting process in the propagation of chaos.

The empirical measure converges to the probability measure concentrated on the law induced by the limiting process given in the propagation of chaos provided the system of processes is exchangeable, see Sznitman \cite{Sznitman1}. In this way we are viewing the empirical measure
\[
\pi^{(n)} := \frac{1}{n}\sum_{i = 1}^n\d_{X_i^{(n)}}
\]
as a map from $\Omega$ to $\mc{P}(C([0, T], \R^d)),$ the space of probability measures on $C([0, T], \R^d)$
equipped with the metric of weak convergence. That is, $\d_{f}$ is the delta mass concentrated on the continuous function $f.$ We call this view of $\pi^{(n)}$ scheme (A), and we will call $\pi^{(n)}$ the \emph{empirical measure}. \\
\\
% There is another natural way of viewing $\pi^{(n)}$, and that is viewing it as a measure-valued process. For fixed $t,$ clearly 
% \[
% \pi^{(n)}_t := \frac{1}{n}\sum_{i = 1}^n\d_{X_i^{(n)}(t)}
% \]
% is a random measure with support in $D.$ Here $\d_x$ is delta mass concentrated at the location $x \in \R^d.$ That is, $\pi^{(n)}_t$ is a random element of $\mc{P}(D)$ equipped with metric of weak convergence. Letting $t$ vary within the interval $[0, T]$, continuity of the processes $X_i^{(n)}(\cdot)$ implies that
% $\{\pi^{(n)}_t : t \in [0, T]\}$ is almost surely a continuous measure valued process (i.e.\ a.s.\ an element in $C([0, T], \mc{P}(D))$). We call this scheme (B), and will call $\pi^{(n)}(\cdot)$ the \emph{empirical process}. One can often show that convergence of the empirical measure in (A) implies convergence of the empirical process in (B) to the density of the limiting process in the propagation of chaos. 
We state the hydrodynamic limit as it applies to both schemes.

\begin{theorem}[HL, Scheme (A)]
	Assume the same setting as in Theorem \ref{th:POC} (propagation of chaos), and let $\wt{X}_1$ be the limiting process in that theorem statement. Consider the empirical measure
	$\pi^{(n)}(\omega) = \frac{1}{n}\sum_{i = 1}^n\d_{X_i^{(n)}(\omega)}.$ Then,
	\[
	\pi^{(n)} \implies \d_{\wt{X}_1}.
	\]
\end{theorem}

\begin{theorem}[HL, Scheme (B)]
	Assume the same setting as in Theorem \ref{th:POC} (Propagation of Chaos), and let $p(t, x)$ denote the transition density of the process $\{\wt{X}_1(t) : t \in [0, T]\}$. Then, $\pi_t^{(n)}$ converges in distribution to $p(t, \md x)$ in the space $C([0, T], \mc{P}(\R))$.
\end{theorem}


%\begin{acknowledgements}
%If you'd like to thank anyone, place your comments here
%and remove the percent signs.
%\end{acknowledgements}


% Authors must disclose all relationships or interests that 
% could have direct or potential influence or impart bias on 
% the work: 
%
% \section*{Conflict of interest}
%
% The authors declare that they have no conflict of interest.

\bibliography{ref_Prop}

% BibTeX users please use one of
%\bibliographystyle{spbasic}      % basic style, author-year citations
\bibliographystyle{spmpsci}      % mathematics and physical sciences
%\bibliographystyle{spphys}       % APS-like style for physics
%\bibliography{}   % name your BibTeX data base


%\bibliographystyle{alpha}

% Non-BibTeX users please use
%\begin{thebibliography}{}
%%
%% and use \bibitem to create references. Consult the Instructions
%% for authors for reference list style.
%%
%\bibitem{RefJ}
%% Format for Journal Reference
%Author, Article title, Journal, Volume, page numbers (year)
%% Format for books
%\bibitem{RefB}
%Author, Book title, page numbers. Publisher, place (year)
%% etc
%\end{thebibliography}

\end{document}
% end of file template.tex

